\documentclass[compressed,dvips,letter]{beamer}

\mode<presentation>
{
  \usetheme{Boadilla}
  \setbeamercovered{dynamic}
}

\input{talk-preamble}
\usepackage{array}
\usepackage{subfigure}
\usepackage{multirow}
\usepackage{arydshln}
\usepackage{verbatim}
\usepackage{graphicx}
\usepackage{amsmath}
%\usepackage{pdfpages}


\setbeamertemplate{navigation symbols}{}
\setlength{\TPHorizModule}{1cm}
\setlength{\TPVertModule}{1cm}


\usefonttheme{professionalfonts}
\title[A Guide to Julia's Dependencies]
{A Guide to Julia's Dependencies\\
\vspace{5pt}
\small aka why does Julia take so long to compile from source?}

\author{Tony Kelman}

\institute[{\fontfamily{pcr}\selectfont @tkelman}]
{
  \footnotesize ({\fontfamily{pcr}\selectfont @tkelman} on Github)
}

\date[BAJU, Dec 4 2014]{\includegraphics[width=100pt]{fig/BAJU_logo.eps}
\\Bay Area Julia Users\\December 4, 2014}

\subject{}


\begin{document}

%\setbeamertemplate{blocks}[rounded][shadow=true]

\setbeamertemplate{footline}{}
\begin{frame}
  \titlepage
\end{frame}


%
%===========================================================================================
%
\setcounter{framenumber}{0}
\setbeamertemplate{footline}[infolines theme]



\begin{frame}[fragile]\frametitle{Who's this guy?}
\begin{itemize}
  \item Grad student at Berkeley in Mechanical Engineering
  \item Julia user and contributor since February 2014
  \item On Github: \includegraphics[width=50pt]{fig/Rejected_avatar.eps} {\tiny (look up ``Rejected" on Youtube)}
  \item Because I was curious, which files have I committed to most?
  
  \tiny \begin{verbatim}
$ git log --author=kelman --pretty=oneline --name-only | sort | uniq -c | sort -r -n | head -n 15
     29 deps/Makefile
     17 Make.inc
     12 Makefile
      8 base/interactiveutil.jl
      6 src/flisp/Makefile
      6 README.md
      5 test/sparse.jl
      5 test/file.jl
      5 src/sys.c
      5 src/support/Makefile
      5 src/debuginfo.cpp
      5 LICENSE.md
      5 contrib/windows/msys_build.sh
      5 appveyor.yml
      5 .travis.yml
  \end{verbatim}
  \only<2>{\normalsize \vspace{-118pt}\alert<2>{\hspace{65pt} $\nwarrow$ \vspace{-8pt} \\ \hspace{77pt}  subject of this talk} \vspace{91.2pt}}

\end{itemize}

\end{frame}
%
%--------------------------------------------------------------
%

\begin{frame}\frametitle{Julia's dependencies}

\begin{block}{Open source and the shoulders of giants}
\footnotesize
``If I have seen further it is by standing on the shoulders of giants." -- Isaac Newton
~ \\
~ \\
``We all build our ideas on the best ideas we can find. Now imagine if there were no more good ideas we were allowed to use." -- Bob Young, co-founder of Red Hat
\end{block}

\begin{itemize}
\item \texttt{git clone git://github.com/JuliaLang/julia.git \\ cd julia \&\& make} ~~ takes a while -- the first time
\item \texttt{deps/Makefile} downloads, configures, and compiles each dependency -- a few of them are much bigger than Julia is
\item Three groups of dependencies:
\begin{enumerate}
\item Linked into \texttt{libjulia}, necessary to run the Julia JIT compiler
\item Used in Julia's standard library via \texttt{ccall()} or process spawning
\item Tools used only at build time to compile Julia (or other deps)
\end{enumerate}

\end{itemize}

What does all this code do?
\end{frame}

%
%--------------------------------------------------------------
%

\begin{frame}\frametitle{Group 1, required for \texttt{libjulia}}
\begin{itemize}
\item LLVM
\begin{itemize}
\item JIT compiler
\end{itemize}
\item libuv
\begin{itemize}
\item Cross-platform input/output
\end{itemize}
\item libunwind
\begin{itemize}
\item Handling backtraces
\end{itemize}
\item utf8proc
\begin{itemize}
\item Unicode processing
\end{itemize}
\end{itemize}
\end{frame}

%
%--------------------------------------------------------------
%

\begin{frame}\frametitle{LLVM, \url{http://llvm.org}}
\begin{minipage}{0.35\textwidth}
\includegraphics[width=125pt]{fig/LLVM_logo.eps}
\end{minipage} \begin{minipage}{0.62\textwidth}
\begin{itemize}
\item What: formerly ``Low Level Virtual Machine,'' today general purpose compiler infrastructure
\item Who: many contributors from Apple, Google, Intel, Mozilla, Julia, etc. \\ Used by Clang, Rust, Swift, Emscripten, WebKit (Safari)
\item When: originally Chris Lattner's Masters and Ph.D theses, circa 2003
\item License: University of Illinois / NCSA (permissive, BSD-style)
\item Written in: \texttt{C++}
\item Use in Julia: just in time compiler
\end{itemize}
\end{minipage}

\end{frame}

%
%--------------------------------------------------------------
%

\begin{frame}\frametitle{A quick introduction to classical compiler design - 1}
From ``The Architecture of Open Source Applications,'' \url{http://www.aosabook.org/en/llvm.html}
\begin{itemize}
\item Basic 3 phase compiler
\begin{figure}\includegraphics[width=250pt]{fig/SimpleCompiler.eps}\end{figure}
\item One input language, one target architecture
\end{itemize}
\end{frame}

%
%--------------------------------------------------------------
%

\begin{frame}\frametitle{A quick introduction to classical compiler design - 2}
\begin{itemize}
\item Modular compiler architecture
\begin{figure}\includegraphics[width=250pt]{fig/LLVMCompiler.eps}\end{figure}
\item Reuse core components across multiple languages and architectures
\item LLVM intermediate representation (IR)
\begin{itemize}
\item Sort of like ``cross-platform assembly''
\item Try out {\fontfamily{pcr}\selectfont @code\_llvm} in Julia
\end{itemize}
\end{itemize}
\end{frame}

%
%--------------------------------------------------------------
%

\begin{frame}\frametitle{libuv, \url{https://github.com/libuv/libuv}}
\vspace{10pt}
\begin{minipage}{0.35\textwidth}
\includegraphics[width=125pt]{fig/libuv_logo.eps}
\end{minipage} \begin{minipage}{0.62\textwidth}
\begin{itemize}
\item What: ``Lib Unicorn Velociraptor''? \\
pun on \texttt{-luv} to link to the library? \\
cross platform asynchronous I/O
\item Who: originally written to port Node.js to Windows, also used by Luvit, Pyuv, formerly Rust
\item When: 2011
\item License: MIT
\item Written in: \texttt{C}
\item Use in Julia: input/output, file system, process spawning, networking, terminal handling, without requiring POSIX
\end{itemize}
\end{minipage}

\vspace{15pt}
Julia uses a minor fork of libuv to support process piping syntax {\fontfamily{pcr}\selectfont run(`ls`|>`sort`)}, need work and pull requests to use upstream

\end{frame}

%
%--------------------------------------------------------------
%

\begin{frame}\frametitle{libunwind, \url{http://nongnu.org/libunwind}}
\begin{itemize}
\item What: Support library for backtraces and profiling on Linux
\item Who: David Mosberger, Arun Sharma, other contributors
\item When: 2002
\item License: MIT
\item Written in: \texttt{C}
\end{itemize}

~\\
~\\
Different library from Apple, written in \texttt{C++} and assembly, used on OSX \url{https://github.com/JuliaLang/libosxunwind}

\end{frame}

%
%--------------------------------------------------------------
%

\begin{frame}\frametitle{utf8proc, {\small \url{http://public-software-group.org/utf8proc}}}
\begin{itemize}
\item What: Unicode processing, normalization, case folding
\item Who: Public Software Group
\item When: 2006
\item License: MIT
\item Written in: \texttt{C}
\end{itemize}


Unicode is complicated!
\begin{itemize}
\item Variable width encodings
\item Combining characters
\item Different normalization forms
\item Similar looking but distinct codepoints
\end{itemize}

~\\
Forked and updated to latest Unicode version by Steven G. Johnson and Jiahao Chen for Julia 0.4-dev, {\footnotesize \url{https://github.com/JuliaLang/libmojibake}}

\end{frame}

%
%--------------------------------------------------------------
%

\begin{frame}\frametitle{Group 2, used in Julia standard library}
\begin{itemize}
\item OpenBLAS: Dense linear algebra
\item SuiteSparse: Sparse matrix factorizations
\item ARPACK: Sparse eigenvalue problems
\item FFTW: Fourier transforms
\item GMP: Arbitrary precision integers (\texttt{BigInt})
\item MPFR: Arbitrary precision floating point (\texttt{BigFloat})
\item PCRE: Regular expressions
\item OpenLibm: Math functions normally found in \texttt{libm}
\item OpenSpecFun: Additional special functions
\item dSFMT: Random number generation
\item Rmath: Distributions and statistical functions
\item double-conversion: Floating point to string conversion (Julia $\leq$ 0.3.x)
\item Git (in Julia $\leq$ 0.3.x) or libgit2 (0.4 WIP): Package manager
\end{itemize}
\end{frame}

%
%--------------------------------------------------------------
%

\begin{frame}\frametitle{OpenBLAS, \url{http://openblas.net}}

\begin{minipage}{0.3\textwidth}
\includegraphics[width=100pt]{fig/LAPACK_logo.eps}
\end{minipage} \begin{minipage}{0.68\textwidth}
\begin{itemize}
\item What: Open source high performance implementation of the Basic Linear Algebra Subprograms (BLAS) and Linear Algebra Package (LAPACK)
\item Who: originally Kazushige Goto, now maintained by Xianyi Zhang, Werner Saar, and others
\item When: GotoBLAS 2002, OpenBLAS 2011
\item License: BSD
\item Written in: Fortran, assembly, \texttt{C}
\item Use in Julia: dense linear algebra
\end{itemize}
\end{minipage}

\vspace{15pt}
Takes a long time to compile when \texttt{OPENBLAS\_DYNAMIC\_ARCH=1}, \\
builds optimized kernel implementations for many recent CPU families \\
(Nehalem, Sandy Bridge, Haswell, etc)

\end{frame}

%
%--------------------------------------------------------------
%

\begin{frame}\frametitle{BLAS in one slide}

\includegraphics[width=0.9\textwidth]{fig/BLAS-snapshot.eps}

\end{frame}

%
%--------------------------------------------------------------
%

\begin{frame}\frametitle{Levels of BLAS and LAPACK}
\begin{itemize}
\item Level 1 BLAS
\begin{itemize}
\item Vector operations: scale, add, copy, dot product, find largest element
\item $O(n)$ operations on $O(n)$ data
\end{itemize}
\item Level 2 BLAS
\begin{itemize}
\item Matrix-vector operations: $A * x$, triangular solve, rank 1 or 2 updates
\item $O(n^2)$ operations on $O(n^2)$ data
\end{itemize}
\item Level 3 BLAS
\begin{itemize}
\item Matrix-matrix operations: $A * B$, triangular solve for multiple right hand sides, rank $k$ updates
\item $O(n^3)$ operations on $O(n^2)$ data
\item Where the important cache optimizations happen
\end{itemize}
\item LAPACK (API does not fit on one slide)
\begin{itemize}
\item Higher level factorizations, eigenvalue and singular value decompositions designed to use efficient BLAS-3 operations
\end{itemize}
\item Reference Fortran implementations from Netlib are slow
\begin{itemize}
\item Better to use an optimized (SIMD, multithreaded) \\
implementation like OpenBLAS or Intel MKL
\end{itemize}
\end{itemize}
\end{frame}

%
%--------------------------------------------------------------
%

\begin{frame}\frametitle{SuiteSparse, \url{http://suitesparse.com}}

\vspace{-5pt}
\begin{center}
\includegraphics[width=0.6\textwidth]{fig/SuiteSparse_logo.eps}
\end{center}
\vspace{-15pt}
\begin{itemize}
\item What: Sparse linear algebra -- LU, Cholesky, and QR factorizations
\item Who: Tim Davis, also used by Matlab, Mathematica, Google Ceres
\item When: 2006 or earlier?
\item License: LGPL
\item Written in: \texttt{C}, \texttt{C++}
\end{itemize}

What makes a matrix sparse?
\begin{itemize}
\item Enough elements are zero to be worth taking advantage of
\item Nonzero structure and permutations very important
\item Graph theory for structure, linear algebra for numerics
\end{itemize}

\end{frame}

%
%--------------------------------------------------------------
%

\begin{frame}\frametitle{ARPACK, {\small \url{https://github.com/opencollab/arpack-ng}}}
\begin{itemize}
\item What: ARnoldi PACKage for eigenvalue problems $A x = \lambda x$ with \\
sparse $A$, iterative algorithms to find a small set of $\lambda$ values
\item Who: Rich Lehoucq, Kristi Maschoff, et al, \\
also used by SciPy, Matlab, Octave
\item When: 1996
\item License: BSD
\item Written in: Fortran 77
\end{itemize}

~\\
Work in progress to replace this with pure Julia code, see \url{https://github.com/JuliaLang/IterativeSolvers.jl/pull/31}

\end{frame}

%
%--------------------------------------------------------------
%

\begin{frame}\frametitle{FFTW, \url{http://fftw.org}}

\vspace{-5pt}
\begin{center}
\includegraphics[width=0.4\textwidth]{fig/fftw-logo-med.eps}
\end{center}
\vspace{-15pt}
\begin{itemize}
\item What: Fastest Fourier Transform in the West
\item Who: Matteo Frigo and Steven G. Johnson
\item When: 1997
\item License: GPL
\item Written in: \texttt{C}, code generator in OCaml
\end{itemize}

~\\
Work in progress to write a pure Julia FFT and move FFTW to an optional package, see \url{https://github.com/JuliaLang/julia/pull/6193}
\end{frame}

%
%--------------------------------------------------------------
%

\begin{frame}\frametitle{GMP, \url{https://gmplib.org}}

\begin{center}
\includegraphics[width=0.4\textwidth]{fig/GMP_logo.eps}
\end{center}
\begin{itemize}
\item What: GNU Multiple Precision library for arbitrary precision integer arithmetic (\texttt{BigInt} in Julia)
\item Who: Torbj\"{o}rn Granlund, GNU Project, many contributors
\item When: 1991
\item License: LGPL
\item Written in: \texttt{C}, assembly
\end{itemize}

\end{frame}

%
%--------------------------------------------------------------
%

\begin{frame}\frametitle{MPFR, \url{http://mpfr.org}}

\begin{center}
\includegraphics[width=0.4\textwidth]{fig/Mpfr_logo.eps}
\end{center}
\begin{itemize}
\item What: GNU Multiple Precision Floating-point Reliable library for arbitrary precision floating-point arithmetic (\texttt{BigFloat} in Julia)
\item Who: INRIA, GNU Project, many contributors
\item When: 2000
\item License: LGPL
\item Written in: \texttt{C}
\end{itemize}

\end{frame}

%
%--------------------------------------------------------------
%

\begin{frame}\frametitle{PCRE, \url{http://pcre.org}}

\begin{block}
\footnotesize
Some people, when confronted with a problem, think ``I know, I'll use regular expressions.'' Now they have two problems. -- Jamie Zawinski
\end{block}

\begin{itemize}
\item What: Perl Compatible Regular Expressions (\texttt{r"..."} in Julia)
\item Who: Philip Hazel, Zoltan Herczeg
\item When: 1997
\item License: BSD
\item Written in: \texttt{C}
\end{itemize}

\end{frame}

%
%--------------------------------------------------------------
%

\begin{frame}\frametitle{OpenLibm, \url{http://openlibm.org}}

\begin{itemize}
\item What: ``high quality, portable, standalone C mathematical library"
\item Who: Viral Shah and other JuliaLang contributors, code originally from FreeBSD msun, OpenBSD libm, and Freely Distributable FDLIBM (\url{http://www.netlib.org/fdlibm})
\item When: 2011, original code from 1992 or earlier
\item License: BSD
\item Written in: \texttt{C}, assembly
\item Why: Performance and accuracy of system \texttt{libm} for trig, sqrt, exp, log, etc varies between platforms, more reliable to build our own
\end{itemize}

\end{frame}

%
%--------------------------------------------------------------
%

\begin{frame}\frametitle{OpenSpecFun, {\small \url{https://github.com/JuliaLang/openspecfun}}}

\begin{itemize}
\item What: Collection of special functions -- Bessel and Airy functions from AMOS library, complex error functions from Faddeeva
\item Who: Donald Amos from \url{http://netlib.org/amos}, \\
Faddeeva by Steven G. Johnson {\small \url{http://ab-initio.mit.edu/wiki/index.php/Faddeeva_Package}}
\item When: Amos code from 1985 or earlier, Faddeeva 2012, OpenSpecFun 2013
\item License: MIT, Public Domain
\item Written in: Fortran 77, \texttt{C++}, \texttt{C}
\item Why: Pieces that were split out from OpenLibm because they are \\
not included in system \texttt{libm} libraries
\end{itemize}

\end{frame}

%
%--------------------------------------------------------------
%

\begin{frame}\frametitle{dSFMT, {\small \url{http://www.math.sci.hiroshima-u.ac.jp/~m-mat/MT/SFMT}}}

\begin{itemize}
\item What: Double precision SIMD-oriented Fast Mersenne Twister random number generator (RNG)
\item Who: Matsuo Saito and Makoto Matsumoto
\item When: 2007
\item License: BSD
\item Written in: \texttt{C}
\end{itemize}

~\\
Julia's RNG code is undergoing heavy development on 0.4-dev right now, good chance dSFMT might be replaced by some alternate RNG \\
see \url{https://github.com/JuliaLang/julia/issues/8786}

\end{frame}

%
%--------------------------------------------------------------
%

\begin{frame}\frametitle{Rmath, \url{https://github.com/JuliaLang/Rmath}}

\begin{itemize}
\item What: Distributions and statistical functions from R, \\
forked and patched to use same dSFMT RNG as Julia
\item Who: R contributors, fork by Viral Shah
\item When: Possibly as old as R, 1997? JuliaLang fork from 2013
\item License: GPL
\item Written in: \texttt{C}
\end{itemize}

~\\
Not used by base Julia, only Distributions.jl and HypothesisTests.jl! \\
see \url{https://github.com/JuliaStats/Distributions.jl/pull/138} for work to remove the dependency

\end{frame}



\end{document}